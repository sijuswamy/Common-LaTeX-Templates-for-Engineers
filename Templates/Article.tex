\documentclass[10pt]{article}
\usepackage{amsmath}
\usepackage{graphicx}
\usepackage{hyperref}
\usepackage{natbib}
\usepackage{geometry}
\geometry{margin=1in}

\title{Your Article Title Here}
\author{First Author \\ \texttt{first.author@example.com} \\ Department of Mathematics, Saintgits University, City, Country \and Second Author \\ Department of Electronics, MIT, City, Country}
\date{\today}

\begin{document}

\maketitle

\begin{abstract}
This is a brief summary of your article. It should clearly outline the main findings and significance of your work.
\end{abstract}

\section{Introduction}
The introduction should provide background information and clearly state the objectives of the research.

\section{Methodology}
Describe the methods used in your research. Include any relevant equations or techniques.

\subsection{Subsection Example}
You can create subsections if needed.

\section{Results}
Present your findings here. Use figures and tables to support your results.

\begin{figure}[htbp]
    \centering
    \includegraphics[width=0.8\columnwidth]{cover.png}
    \caption{Caption for your figure.}
    \label{fig:example}
\end{figure}

\section{Discussion}
Interpret your results and discuss their implications. Compare with previous work and highlight the significance of your findings.
A sample reference of article/ book is \cite{goossens1994latex}
\section{Conclusion}
Summarize the main points of your article and suggest possible future research directions.

\section*{Acknowledgments}
Acknowledge any assistance or funding received during your research.
\bibliographystyle{plain} % Change to your preferred style
\bibliography{reference.bib} % Add your .bib file here


\end{document}
